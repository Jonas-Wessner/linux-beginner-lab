\section{Process Management}

A process is an instance of an executing program. For example \lstinline{firefox} or \lstinline{ls} are programs. Although \lstinline{firefox} is only installed once on a computer, one can open multiple windows of \lstinline{firefox}, which are independent running instances of the program \textendash{} independent processes.

The terminal allows the user to execute programs as foreground or as background processes. All commands shown so far have been executed as foreground processes. Foreground processes block the shell they are running in as long as the foreground process runs. For the commands shown so far that is not an issue, as they complete in a couple of milliseconds. However, there are tasks e.g. installing programs, which may take a lot longer. In case a user wants to install five programs, of which each takes three minutes to install, he can start all installations as background processes, which means they all run in parallel and do not block the current shell. That way he can continue using the current shell for other things in the meantime.

In order to execute a command as a background process, an ampersand has to be appended to the command. For example the command \lstinline{sleep 60 &} executes the \lstinline{sleep 60} command, which lets the current shell do nothing for 60 seconds, in the background. After a background process has been started, its status can be checked using the \lstinline{jobs} command, which shows all running or stopped background processes. In order to pull a process from the background back into the foreground, the command \lstinline{fg <id>} (\textbf{f}ore\textbf{g}round) can be used. Running foreground processes can be stopped (not canceled) by pressing the Ctrl+Z. The \lstinline{jobs} command then shows them as stopped processes. In order to resume a process in the foreground, the \lstinline{fg <id>} command can be used as well. In order to resume it in the background, the \lstinline{bg <id>} (\textbf{b}ack\textbf{g}round) command can be used. To cancel a foreground process completely, one can press Ctrl+C.