\section{Subshells}

The term subshell describes the execution of a command in another shell inside of the current shell. This can be useful, if the result of one command is required as the argument to another command. A subshell is created by a dollar sign followed by a pair of brackets encapsulating the command to be executed in the subshell like so: \lstinline{$(subcommand)}. The shells are interpreted from inside out starting from the most inner subshell. After each execution of a subshell, its code is replaced by the content of the standard output stream of that subcommand and is then subject to execution of the next higher order shell. As a very simple example, one could use the command \lstinline{echo "Today is $(date)"} to print a text which says what todays date is. First, the subshell will be executed, which results in the current date and time. Then the outer command will be executed, which takes the text as an argument, which contains the current date. Another example would be the command \lstinline{mkdir $(pwd)/new_dir}, which creates a directory called \lstinline{new_dir} inside of the home directory of the user who executes the command. Of course, this could also be achieved by using the tilde symbol (\textasciitilde) as a replacement of the users home directory.

Advanced users can use subshells to create powerful functionalities, especially as a part of scripting. These are not shown in this document, but are based on the same principles shown above.