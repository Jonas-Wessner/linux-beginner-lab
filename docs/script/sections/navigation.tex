\section{Navigation}
\label{sec:navigation}

The shell always has a current position, which can be printed using the \lstinline{pwd} (\textbf{p}rint \textbf{w}orking \textbf{d}irectory) command. When first opening the shell, the position is the home directory of the current user. On a system there can exist multiple users, which are essentially accounts. Each user usually has his own home directory inside of the directory \lstinline{/home}. Apart from the normal users, there also exists a user called \lstinline{root}, who is a system user and has privileged access on the system. For security reasons some data on the system can only be modified by the \lstinline{root} user.

The file system on a linux computer is organized as a tree. The root of the tree is called \lstinline{/} and is the upper most directory. The \lstinline{cd <dest>} (\textbf{c}hange \textbf{d}irectory) command can be used, to navigate through the directory tree. The argument \lstinline{<dest>} is the destination file path to navigate to. A file path is generally speaking a position in the directory tree pointing to either a directory or a file, which is formed by the names of directories separated by forward slashes. There are two types of file paths: absolute paths and relative paths. Absolute paths start at the root of the directory tree and are therefore independent from the current position of the shell. An example would be \lstinline{/home/myuser/example.txt}. Relative paths start either with a word, a dot, or a double dot. The dot in file paths is a replacement for the current working directory i.e. the result of the command \lstinline{pwd}. Therefore the above file path could also be written as \lstinline{./myuser/example.txt}, if the current working directory is \lstinline{/home}. Also, generally \lstinline{bash} assumes a relative path if the first part of a file path is a word. Therefore, the above path could also be simplified to \lstinline{myuser/example.txt}. Moreover, because the home directory is an often used part of file paths, the character \textasciitilde{} can be used as a replacement for the users home directory. Therefore, if the current user is \lstinline{myuser}, the file path can be further simplified to \lstinline{~/example.txt}. A double dot in a file path is used to navigate one directory upwards. For example, if the current position of the shell is \lstinline{/home}, then the file path \lstinline{../bin} would point to \lstinline{/bin}.

All contents of the current directory can be printed using the command \lstinline{ls} (\textbf{l}i\textbf{s}t). The contents of a directory can be files and other directories. The behavior of the \lstinline{ls} command can be modified by additional options: \lstinline{ls -1} prints all entries on separate lines, \lstinline{ls -l} (\textbf{l}i\textbf{s}t \textbf{l}ong) prints additional information for each entry. There are many other options for \lstinline{ls}, of which some will be discovered in later sections. Many programs accept additional modifying parameters just like \lstinline{ls} does. It is a convention that those options usually start with a hyphen or a double hyphen.

In order to type commands faster, one can use the tab key to complete parts of the command. This is especially helpful when typing long directory names. If there is no unambiguous completion for the command part, one can press the tab key twice to receive a list of all possibilities to complete the command.

Previously used commands can be reused in order to avoid typing the same command over and over again. In order to do that one can use the up and down arrows to navigate through \lstinline{bash}s command history. Furthermore, the complete history can be printed with the command \lstinline{history}.