\section{Streams and Redirection}

Every program has 3 streams of information: the standard input stream \lstinline{stdin}, the standard output stream \lstinline{stdout} and the standard error stream \lstinline{stderr}. Streams are a common  way of a program to communicate with other programs or interact with the user. If required, programs read from \lstinline{stdin} to receive input from the user. If a program wants to emit some output, it writes that to \lstinline{stdout}. Error messages are written to \lstinline{stderr}.

The streams can be connected to other programs. For example, \lstinline{stdout} and \lstinline{stderr} of each program running in the terminal are by default connected to the shell, which then displays their content to the user. E.g. \lstinline{ls} writes its output to its standard output stream, which the terminal then prints to the computers display. Also \lstinline{stdin} of any program running in the terminal is by default connected to the computers keyboard. Therefore, when a program expects input, it can read whatever the user types in.

Sometimes it can be useful to redirect the stream of a program to somewhere else than the terminal. Using the greater-than-symbol (\lstinline{>}), the standard output stream of a command can be written to a file. For instance, to save the names of the current directories contents to a file, one could use the command \lstinline{ls > result.txt}. In order to not overwrite the contents of a file, but instead append to the file, two greater-than-symbols (\lstinline{>>}) must be used. Similarly, the less-than-symbol (\lstinline{<}) can be used to connect the content of a file to the standard input stream of a program. As an example, The \lstinline{egrep} program reads from its standard input stream and searches for lines containing a given text. The command \lstinline{egrep "abc" < file.txt} would connect the contents of the file to the input stream of the \lstinline{egrep} program and therefore result in searching for the text \quotes{abc} in the file called \lstinline{file.txt}. Furthermore, it is possible to connect the standard output stream of a program to the standard input stream of another program. This is called piping and requires the use of the symbol \lstinline{|}. For instance, to search for zip-files in the current directory, one could use the command \lstinline{ls | grep ".zip"}. This command pipes the standard output of the \lstinline{ls} command to the input stream of the \lstinline{grep} command, which then searches for the given pattern in its input stream.
