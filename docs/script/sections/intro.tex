\section{Introduction}

The command line (also called terminal, console or shell) is a tool to control a computer merely by typing in text and receiving text as a response. The disadvantage of this is that it is unintuitive for many users when encountering it for the first time. However, on the other side, it also is more powerful than graphical user interfaces and therefore very commonly used in the context of computer science. One of the advantages is that it is available on every computer. On the contrary, a desktop with a graphical user interface (GUI) often is not available on servers or similar machines, which are targeted on high performance. Another reason is, that for programmers it is much easier to develop programs for the command line compared to GUI applications, because of the extra layer of complexity, which comes with the design of a user interface. Therefore, many applications in the context of computer science only have a command line interface (CLI). Another extremely big advantage of the command line is that it can be used to automate tasks by combining multiple commands into a script, whereas with GUIs there is a human required who clicks buttons.

Every operating system has a terminal. However, we will be focusing on the Linux terminal here, which defaults to \lstinline{bash}. \lstinline{bash} is the name of the program which interprets the commands you type in. There are also other terminals, but \lstinline{bash} is by far the most common one.

You may also ask yourself why computer scientists tend to choose Linux over other operation systems. There are many reasons for this, of which some will be presented in the following.

\begin{itemize}
    \item It is free to use and open source, which makes users independent from companies.
    \item It is more performant (i.e. faster) than e.g. Windows, which is one of the reasons pretty much all of the servers, computer clusters and supercomputers run Linux.
    \item Therefore, a lot of software is targeted on Linux as well.
    \item It tends to be more secure than other operating systems.
    \item In Linux the user has full control over system updates, whereas with Windows updates might be forced.
\end{itemize}

